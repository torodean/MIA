\chapter{Sequencer} \label{sequencer}\index{sequencer}
\pagestyle{fancy}

The sequencer is a sub program built into MIA for performing key/button simulation sequences. This can be used for Performing mundane tasks automatically and repeatable tasks on a timer while away from the computer. More advanced uses can also be for botting and other repeatable tasks.

\section{Using the Sequencer}

The sequencer acquires its functionality from the MIASequences.txt file. The MIASequences.txt file will appear similar to the following.

\begin{lstlisting}
#============================================================================
# Name        : MIASequences.txt
# Author      : Antonius Torode
# Date        : 12/26/2019
# Copyright   : This file can be used under the conditions of Antonius' 
#				 General Purpose License (AGPL).
# Description : MIA combinations for button sequences.
#============================================================================

# This file is formatted similar to the MIAConfig file.
# Create a commented line using the '#' character. 
# Comments must be on their own line.
# This file must be of the proper format to work with MIA.
# Create a combination name using 'SEQUENCENAME=name'.
# Define the timing between sequence parameters with 'timing=3000', units are milliseconds.
# After declaring a name for the command sequence, define the command sequence using the following.
# XXXX,YYYY=LEFTCLICK
# type=abc
# The first performs a left click at some coordinate (XXXX,YYYY).
# The second simply types 'abc'.
# Do not include spaces unless within a string variable. 
# Actions and program variables should be capitalized.
# The sequence name must be defined at the start of a sequence.
# The end of a sequence must be defined by ENDOFSEQUENCE.

#Sequence definitions below...
\end{lstlisting}

The sequencer will run sequence definitions from the sequences file simply by typing the \inlinecode{sequencer} command in MIA. The MIA program will then ask for a sequence name to run. The names are defined when the sequence is created based on the sequencer file input.

\section{Defining a Sequence}

A sequence is defined by first creating a sequence name using the \inlinecode{SEQUENCENAME} keyword and ending with the \inlinecode{ENDOFSEQUENCE} keyword. The following example will demonstrate all of the valid commands in the sequencer file. The descriptions of what each line do are below.

\begin{lstlisting}
# This combination is for testing.
SEQUENCENAME=test
TIMING=3000
HOVERTIME=2000
145,887=LEFTCLICK
219,889=HOVER
TYPE=abcd
145,887=LEFTCLICK
ENDOFSEQUENCE
\end{lstlisting}

\begin{enumerate}
	\item \inlinecode{SEQUENCENAME=test} This line creates a sequence with the name ``test''.
	\item \inlinecode{TIMING=3000} This sets the timing between each event in the sequence. The units are in milliseconds.
	\item \inlinecode{HOVERTIME=2000} This defines the time to hover when the hover keyword is used.
	\item \inlinecode{145,887=LEFTCLICK} This is a command signaling to move the mouse to the coordinates (145,887) and then perform a LEFTCLICK.
	\item \inlinecode{219,889=HOVER} This is a command signaling to move the mouse to the coordinates (219,889) and then perform the hover command.
	\item \inlinecode{TYPE=abcd} This command will not move the mouse, but rather type the text entered. In this case ``abcd'' will be typed.
	\item \inlinecode{ENDOFSEQUENCE} This ends the sequence and prepares the sequencer for a new defined sequence.
\end{enumerate}

\section{Notes on Using The Sequencer}

At the time of writing this, the sequencer is completely new and yet to be fully tested. It will continue to be improved as it is used. It is important to follow the scheme above for defining sequences above closely as errors and bugs are not yet determined. It is also important that all needed parameters be defined properly and no sequences have a duplicate name. The program is not programmed to handle this as of yet.