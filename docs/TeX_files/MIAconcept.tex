\chapter{Multiple Integrated Applications (MIA)}
\pagestyle{fancy}

\section{Introduction}

MIA is designed to be a collection of scripts, tools, programs, and commands that have been created in the past and may be useful in the future. It's original idea was a place for the original author to combine all of his previous applications and codes into one location that can be compiled cross platform. MIA is written in C++ but will contain codes that were originally designed in C\#, Java, Python, and others. MIA is created for the authors personal use but may be used by others if a need or desire arises under the terms of Antonius’ General Purpose License (AGPL). 

The MIA acronym was created by the original author for the sole purpose of this application. The design of MIA is a terminal prompt that accepts commands. There are no plans to convert MIA into a GUI application as there is currently no need; however, some elements may be programmed in that produce a GUI window for certain uses such as graphs. The MIA manual is designed to be an explanation of what MIA contains as well as a guide of how to utilize the MIA program to it's fullest. 

As MIA is continually under development, this document is also. Due to this, it may fall behind and become slightly outdated as I implement and test new features into MIA. I will attempt to keep this document up to date with all of the features MIA contains but I can only do so if time permits.


















\section{MIA Features}

\subsection{MIA Application Framework}

One of the main features of MIA is the core framework and internal setup. The MIA project's core framework and modular architecture form the backbone of its design, providing a robust, extensible, and developer-friendly ecosystem for building and managing applications. This framework enforces a standardized application life cycle, ensures clear dependency management, and integrates seamlessly with supporting systems such as error handling, configuration, and logging. By leveraging modern C++20 features, the core framework promotes type-safe, maintainable, and scalable application development, making it a cornerstone feature of MIA.

The core framework is built around several interconnected components that collectively enhance modularity and usability:
\begin{itemize}\itemsep0em
	\item \textbf{Layered Dependency Structure:} The project is organized into distinct layers---Core, Utilities, Libraries, and Applications---to ensure a unidirectional dependency flow. The Core layer provides fundamental types, error codes, and logging interfaces with no external dependencies. Utilities depend on Core, offering reusable functions like file and string operations. Libraries build upon Core and Utilities for domain-specific logic, such as mathematical functions, while Applications integrate all layers to create executable programs. This structure prevents circular dependencies, enhancing maintainability and scalability. See section \ref{sec:dependancy-structure} for more information.
	\item \textbf{Application Framework:} The framework defines a consistent life cycle for applications, requiring \texttt{initialize} and \texttt{run} methods, enforced at compile-time using C++20 concepts (\texttt{AppInterface}). The \texttt{MIAApplication} base class standardizes command-line argument parsing (e.g., \texttt{-v} for verbose, \texttt{-h} for help) and provides a template function (\texttt{runApp}) and macro (\texttt{MIA\_MAIN}) to streamline application entry points, reducing boilerplate code and ensuring uniform behavior. See section \ref{sec:app-framework} for more information.
	\item \textbf{Supporting Systems:} The framework integrates with robust subsystems:
	\begin{itemize}\itemsep0em
		\item \textbf{Error Handling:} The \texttt{MIAException} class and \texttt{ErrorCode} enumeration provide structured error propagation and descriptive reporting (see section \ref{sec:error-handling}).
		\item \textbf{Configuration:} The \texttt{MIAConfig} class supports multiple formats (e.g., key-value, raw lines) with typed accessors and extensible design (see section \ref{sec:configuration-system}).
		\item \textbf{Logging:} The \texttt{Logger} class and free functions enable thread-safe logging to customizable files with optional verbose output (see section \ref{sec:logging-Framework}).
		\item \textbf{Global Constants and Paths:} The \texttt{constants} and \texttt{paths} namespaces centralize version information and resource locations, adapting to installed or repository-based environments (see Section~\ref{sec:global-constants}).
	\end{itemize}
	\item \textbf{Build and Deployment:} A flexible Bash script (\texttt{build.sh}) and CMake integration automate building, installing, and managing dependencies, supporting platform-specific configurations (primarily Linux, with planned expansions) and release builds (see Sections \ref{sec:build-script} and \ref{sec:CMake-setup}).
\end{itemize}

\subsubsection{Benefits}
The core framework offers several advantages for developers and end-users:
\begin{itemize}\itemsep0em
	\item \textbf{Modularity and Maintainability:} The layered architecture ensures clear separation of concerns, making it easier to modify or extend individual components without affecting others.
	\item \textbf{Type Safety and Consistency:} C++20 concepts enforce interface compliance at compile-time, reducing runtime errors and ensuring uniform application behavior.
	\item \textbf{Developer Productivity:} Standardized interfaces, macros, and automated build tools minimize boilerplate code and streamline development workflows.
	\item \textbf{Extensibility:} The framework's design supports adding new features, configuration types, and platform support with minimal changes to existing code.
	\item \textbf{Robustness:} Integrated error handling, logging, and configuration systems provide reliable diagnostics and flexible runtime behavior.
\end{itemize}

\subsubsection{Summary}
The MIA core framework and modular architecture provide a powerful foundation for building scalable, maintainable applications. By combining a layered dependency structure, a standardized application life cycle, and robust supporting systems, it enables developers to create feature-rich tools with minimal overhead while ensuring reliability and extensibility for future growth.





























\section{Repository Overview}

The current project repository can be found at the following link:
\begin{center}
	\url{https://github.com/torodean/MIA}
\end{center}
This repository is a modern recreation of the original MIA (Multiple Integrated Applications) project available at \url{https://github.com/torodean/Antonius-MIA}. It serves as a centralized platform for storing various utilities, scripts, and applications developed over time, allowing easy reuse and accessibility. The project is modular and designed for cross-platform compatibility.

For a comprehensive list of commands, features, and usage guidelines, refer to the MIA manual located in this file or accessible online at \url{https://github.com/torodean/MIA/docs/}.

\subsection{Project Structure}

\begin{itemize}\itemsep0em
	\item \texttt{bin/} \hfill \\ 
	Contains the main source code for all integrated applications and core libraries, organized by module and functionality.
	
	\item \texttt{docs/} \hfill \\ 
	Houses all project documentation, including manuals, design notes, and usage instructions.
	
	\item \texttt{scripts/} \hfill \\ 
	Stores utility scripts for tasks such as building, installing, and managing the project.
	
	\item \texttt{build/} (generated) \hfill \\ 
	Temporarily holds build artifacts and platform-specific outputs during the compilation process.
	
	\item \texttt{release/} (generated) \hfill \\ 
	Contains release-ready binaries and configuration files for distribution or standalone execution.
	
	\item \texttt{resources/} \hfill \\ 
	Includes static resources and configuration files required by MIA tools and executables.
\end{itemize}

Additional directories may be introduced as the project evolves to support new functionality or organizational needs.
