\chapter{Multiple Integrated Applications (MIA)}
\pagestyle{fancy}

\section{Introduction}

MIA is designed to be a collection of scripts, tools, programs, and commands that have been created in the past and may be useful in the future. It's original idea was a place for the original author to combine all of his previous applications and codes into one location that can be compiled cross platform. MIA is written in C++ but will contain codes that were originally designed in C\#, Java, Python, and others. MIA is created for the authors personal use but may be used by others if a need or desire arises under the terms of Antonius’ General Purpose License (AGPL). 

The MIA acronym was created by the original author for the sole purpose of this application. The design of MIA is a terminal prompt that accepts commands. There are no plans to convert MIA into a GUI application as there is currently no need; however, some elements may be programmed in that produce a GUI window for certain uses such as graphs. The MIA manual is designed to be an explanation of what MIA contains as well as a guide of how to utilize the MIA program to it's fullest. 

As MIA is continually under development, this document is also. Due to this, it may fall behind and become slightly outdated as I implement and test new features into MIA. I will attempt to keep this document up to date with all of the features MIA contains but I can only do so if time permits.





\subsection{Repository Overview}

This repository is a modern recreation of the original MIA (Multiple Integrated Applications) project available at \url{https://github.com/torodean/Antonius-MIA}. It serves as a centralized platform for storing various utilities, scripts, and applications developed over time, allowing easy reuse and accessibility. The project is modular and designed for cross-platform compatibility.

For a comprehensive list of commands, features, and usage guidelines, refer to the MIA manual located in this file or accessible online at \url{https://github.com/torodean/MIA/docs/}.

\subsubsection*{Project Structure}

\begin{itemize}
	\item \texttt{bin/} \hfill \\ 
	Contains the main source code for all integrated applications and core libraries, organized by module and functionality.
	
	\item \texttt{docs/} \hfill \\ 
	Houses all project documentation, including manuals, design notes, and usage instructions.
	
	\item \texttt{scripts/} \hfill \\ 
	Stores utility scripts for tasks such as building, installing, and managing the project.
	
	\item \texttt{build/} (generated) \hfill \\ 
	Temporarily holds build artifacts and platform-specific outputs during the compilation process.
	
	\item \texttt{release/} (generated) \hfill \\ 
	Contains release-ready binaries and configuration files for distribution or standalone execution.
	
	\item \texttt{resources/} \hfill \\ 
	Includes static resources and configuration files required by MIA tools and executables.
\end{itemize}

Additional directories may be introduced as the project evolves to support new functionality or organizational needs.
