\chapter{Coding Standards}

\textbf{IMPORTANT:} These coding standards are still being developed and not finished or finalized. They can safely be followed, but some parts may change, or additions will be made in the future.

\section{Overview}

As of MIA version 2\footnote{Note: Although the coding standards are adopted in version 2, not all of the code yet conforms to this standard. Much of the old code which was migrated over still needs improvements and fixes in order to fit these standards.}, a set of coding standards has been adopted by this project. To maintain consistency, readability, and maintainability across the project, a unified coding standard for all C++, CMake, and Bash code is used. This section outlines the conventions and best practices to be followed by all current and future contributors. Adhering to these guidelines helps reduce bugs, ease code reviews, and improve collaboration if MIA ever became a multi-developer environment.

Since this project was (and still is at the time of writing this) solely for personal edification, learning, and experimentation, its primary purpose has been to explore new tools, techniques, and patterns across C++, CMake, and Bash. However, it also serves as a practical demonstration of my coding style and development practices - much like a portfolio, though firmly within the realm of hobby code. As such, while the codebase may not meet production-level rigor in every detail, this coding standards documentation exists to highlight the conventions I choose to follow, the rationale behind them, and how they reflect what I believe to be a synthesis of the best practices across the diverse landscape of existing standards. This includes adapting principles from well-known style guides while also incorporating personal preferences that I’ve found to improve clarity, maintainability, and efficiency through hands-on experience.

\section{C++ Coding Standards}

\subsection{Formatting and Style}
\begin{itemize}\itemsep0em
    \item \textbf{Indentation}: Use \texttt{spaces} with a width of \texttt{4} spaces.
    \item \textbf{Line length}: Limit lines to \texttt{120} to \texttt{140} characters. Traditionally, less characters are used. However, with newer computers and displays, this rarely makes sense. With a \texttt{120} to \texttt{140} width, most modern-sized computer displays can display 2-3 columns/files of code at a time wth an additional file-browser window. The range in values comes from a standput of developer discretion. Often times, a hard cap at a set line length can produce some fairly unappealing code and less readable, where the, range allows the developer to use discretion based on the code itself. 
    \item \textbf{Brace placement}: \texttt{Allman}. Although the original MIA project was in K\&R format, I've chosen to move to \texttt{Allman} as of version 2. The benefits of Allman include the following:    
    \begin{itemize}\itemsep0em
        \item Aligned Blocks: With Allman, readability is often enhanced because you do not have to visually search the end of variable-length lines for brackets or rely on tabs. Allman can help reduce cognitive load when scanning code. All brackets are at the start of the line and aligned with the ending bracket. 
        \item Clearly Separated Blocks: This style makes each block of code more clearly separated.
        \item Commenting: The initial bracket of a code block can serve as a perfect spot for a brief comment related to the block itself. 
    \end{itemize}
    \item \textbf{Naming conventions}:
    \begin{itemize}\itemsep0em
        \item Classes and structs: \texttt{PascalCase}
        \item Enums and typedefs: \texttt{PascalCase}
        \item Functions and methods: \texttt{camelCase}
        \item Variables: \texttt{camelCase}
        \item Constants: \texttt{ALL\_CAPS} 
    \end{itemize}
\end{itemize}

\subsection{Code Organization and Practices}
\begin{itemize}\itemsep0em
    \item Header files should include include guards or \texttt{\#pragma once}.
    \item Prefer smart pointers over raw pointers where ownership is involved. Smart pointers such as \texttt{std::unique\_ptr} and \texttt{std::shared\_ptr} manage memory automatically and help prevent common issues like memory leaks, dangling pointers, and double deletions. They make ownership semantics explicit and integrate with RAII (Resource Acquisition Is Initialization), ensuring that resources are properly released when no longer needed. Raw pointers can still be used for non-owning references, but ownership should be clearly represented through smart pointers to improve code safety and maintainability.    
    \item Avoid using \texttt{using namespace std;} in headers. Placing \texttt{using namespace} directives in headers pollutes the global namespace for any file that includes that header, potentially causing name collisions and ambiguous references. This can lead to subtle and hard-to-debug errors, especially in large codebases or when integrating with third-party libraries. Instead, fully qualify names (i.e., \texttt{std::vector}) in headers to keep dependencies explicit and avoid unintended side effects.
    \item Separate declaration and implementation: use headers (\texttt{.hpp}) for interfaces and source files (\texttt{.cpp}) for implementation.
    \item Prefer forward declarations over includes in headers when only a pointer or reference to a type is needed. Only include full headers when access to the complete type is required.
    \item Where applicable, default fallback values (i.e., 0, "", false) may be returned in lieu of exceptions for non-critical failures, provided this behavior is explicitly documented. This includes all get*() functions for configuration access.
    \item Exceptions should be used only for unrecoverable errors or contract violations, and custom exceptions (i.e., MIAException) should encapsulate error context where possible.
    \item Use Resource acquisition is initialization (RAII) for resource management.
    \item Use const correctness to express immutability.
    \item Document all public interfaces with Doxygen-style comments.
\end{itemize}



\subsection{Virtual Classes, Abstractions, and PIMPL}

Object-oriented abstractions such as virtual classes and the PIMPL (Pointer to Implementation) idiom are powerful tools but must be used judiciously to avoid unnecessary complexity and indirection. Code should prioritize readability and simplicity over overengineering. Clear, straightforward implementations are easier to maintain, debug, and extend than overly abstract or generalized solutions. Avoid premature optimization or architectural complexity unless justified by concrete requirements. The goal is to make the intent and behavior of the code immediately understandable to other developers. When in doubt, favor code that communicates clearly over code that attempts to be overly clever or flexible without a proven need.

\subsubsection*{Virtual Classes and Abstract Interfaces}

Abstract classes (i.e., classes with pure virtual functions) should be used to define clean, minimal interfaces that enable polymorphism and decouple high-level logic from specific implementations. These interfaces are particularly appropriate in the following cases:
\begin{itemize}\itemsep0em
    \item Multiple interchangeable implementations are required.
    \item Code must be testable via mocks or stubs.
    \item A plugin or extensibility mechanism is being designed.
\end{itemize}
Avoid over-engineering by introducing interfaces prematurely. When there is no clear need for polymorphism, prefer concrete classes for simplicity and clarity. Overall, free-functions are preferred where applicable.

\subsubsection*{PIMPL Idiom}

The PIMPL idiom may be used to encapsulate implementation details and reduce compilation dependencies, particularly for public interfaces that are expected to remain stable while their internals may change. Its use must be justified by the need to isolate dependencies, improve build performance, or enforce ABI stability. The PIMPL (Pointer to Implementation) idiom is used to:
\begin{itemize}\itemsep0em
    \item Hide private data members and internal implementation details.
    \item Reduce build dependencies and improve compilation times.
    \item Achieve binary interface stability across library versions.
\end{itemize}

PIMPL should only be used when these benefits are specifically needed. It introduces runtime indirection and heap allocation, which may be unnecessary in simpler or performance-sensitive components.
\begin{lstlisting}[style=cppstyle]
/**
 * @brief A class demonstrating the PIMPL idiom for encapsulation.
 */
class MyClass 
{
public:
    /**
     * @brief Constructs a MyClass instance.
     */
    MyClass();
    
    /**
     * @brief Destroys the MyClass instance, freeing resources.
     */
    ~MyClass();
private:
    class Data; ///< Forward declaration of implementation details.
    std::unique_ptr<Data> data; ///< Pointer to implementation.
\end{lstlisting}

\paragraph{Usage Guidelines}

\begin{itemize}\itemsep0em
    \item Use virtual functions only where dynamic behavior is explicitly required.
    \item Keep interfaces small, focused, and consistent.
    \item Use PIMPL for library boundaries or complex internals where ABI stability or encapsulation justifies the added indirection.
    \item Avoid using PIMPL as a default or substitute for proper modular design.
\end{itemize}

By adhering to these principles, the code remains both flexible and maintainable without incurring the overhead of unnecessary abstraction.



\subsection{Attributes, Specifiers, and Member Annotation Standards}

\subsubsection*{C++ Attribute Specifiers}

C++ standard attributes such as \texttt{\textbf{[[deprecated]]}}, \texttt{\textbf{[[maybe\_unused]]}}, and others should be used appropriately to enhance clarity, compiler diagnostics, and maintainability.
\begin{itemize}\itemsep0em
  \item \textbf{\texttt{[[deprecated]]}} \\
  Use to mark deprecated interfaces explicitly. Always include an explanatory message:
  \begin{lstlisting}[style=cppstyle]
[[deprecated("Use newMethod() instead")]]
void oldMethod();
  \end{lstlisting}

  \item \textbf{\texttt{[[maybe\_unused]]}} \\
  Apply to functions or variables that may be unused in some contexts (i.e., for debugging or optional features):
  \begin{lstlisting}[style=cppstyle]
[[maybe_unused]]
void dumpConfigMap(std::ostream& os = std::cout) const;
  \end{lstlisting}

  \item \textbf{\texttt{[[nodiscard]]}} \\
  Use for return values where the result must be used by the caller (i.e., for error codes or ownership types).
\end{itemize}
Use attribute specifiers only when their intended behavior is part of the API contract or serves a specific, useful purpose. Avoid excessive or speculative use.

\subsubsection*{\texttt{override} and \texttt{final} Usage}

All overridden virtual methods must be explicitly marked with \texttt{override}. Do \textbf{not} use the \texttt{virtual} keyword on the overriding declaration.
\begin{itemize}\itemsep0em
  \item Correct:
  \begin{lstlisting}[style=cppstyle]
void onStart() override;
  \end{lstlisting}
  \item Incorrect:
  \begin{lstlisting}[style=cppstyle]
virtual void onStart() override;
  \end{lstlisting}
\end{itemize}
Use \texttt{final} on classes or methods where further subclassing or overriding is explicitly disallowed:
\begin{lstlisting}[style=cppstyle]
class Logger final { ... };
void onShutdown() final;
\end{lstlisting}


\subsection{Comments and Documentation}

All code elements including classes, namespaces, methods, typedefs, variables, and files must have Doxygen comments. These comments should primarily be placed in header files to serve as the main documentation source. For methods not declared in headers but implemented in \texttt{.cpp} files, corresponding Doxygen comments should be added in the implementation files. In-code comments must be sufficiently informative to allow any developer to understand the purpose, usage, and modification requirements of the code without needing to consult external documentation. This includes describing the intent of classes and methods, explaining parameters and return values, and clarifying design decisions where applicable.

In \texttt{.cpp} implementation files, comments should be added at the developer’s discretion but are ideally included:
\begin{itemize}\itemsep0em
    \item Before blocks of code that perform distinct tasks,
    \item When specific literal values or “magic numbers” are used for particular reasons,
    \item Where behavior might not be immediately clear from the code itself,
    \item To highlight important side-effects or assumptions.
\end{itemize}
For ease of navigation and maintenance, closing braces of namespaces, classes, and long code blocks should be followed by simple trailing comments indicating their scope, for example:
\begin{lstlisting}[style=cppstyle]
} // namespace myNamespace
\end{lstlisting}
This convention helps match braces quickly without excessive scrolling.

\subsubsection*{Additional guidelines}
\begin{itemize}\itemsep0em
    \item Avoid redundant comments that restate obvious code.
    \item Use complete sentences, present tense, punctuation, and consistent phrasing for clarity.
    \item Use consistent terminology and formatting in all comments.
    \item Prefer clarity and brevity over verbosity.
    \item Update comments promptly alongside code changes to keep them accurate.
    \item All functions that may throw exceptions must document them using @throws. 
\end{itemize}

\subsubsection*{Example Doxygen comments}

For multi-line comments, use the \texttt{/** .. */} comment block.
\begin{lstlisting}[style=cppstyle]
/**
 * @brief Represents a configurable widget.
 * This class manages widget configuration and state.
 * @param config[Config] - Configuration settings for the widget.
 * @return Widget instance initialized with given config.
 */
class Widget 
{
public:
    /**
     * @brief Initializes the widget with given settings.
     * @param settings[Config] - The configuration to apply.
     */
    void initialize(const Config& settings);
};
\end{lstlisting}
For single-line comments that document member variables, use \texttt{///} directly above the definition. These comments should be clear, concise, and written in full sentences when appropriate.
\begin{lstlisting}[style=cppstyle]
/// The name of the configuration file for this object.
std::string configFileName;
\end{lstlisting}
\textbf{Guidelines:}
\begin{itemize}\itemsep0em
    \item Use \texttt{///} rather than \texttt{//} for documentation intended to be parsed by tools like Doxygen.
    \item Comments must describe the purpose or semantics of the variable, not its type or structure (which should be clear from the code).
    \item Maintain grammatical consistency and use proper punctuation.
\end{itemize}
Each enumeration should be preceded by a Doxygen-style comment block explaining its purpose. Individual enum values must use \texttt{///<} inline comments to document their meaning. This ensures that documentation tools capture the purpose of each constant clearly, and that readers can understand the intent without leaving the context of the declaration.
\begin{lstlisting}[style=cppstyle]
/**
 * These are the various actions which the sequencer supports.
 */
enum SequenceActionType
{
    UNKNOWN,     ///< Unknown action - do nothing.
    TYPE,        ///< This will type a sequence of characters.
    SLEEP,       ///< This will wait/pause some time.
    DELAY,       ///< This is the time to wait between each action.
    MOVEMOUSE,   ///< This will move the mouse to a specific coordinate.
    CLICK        ///< This will perform a click with the mouse.
};
\end{lstlisting}
\textbf{Guideline:} Always use \texttt{///<} on the same line as the enum value, not on a new line above. Comments should be concise, use present tense, and describe the runtime behavior.


\subsubsection*{Line Comments in Implementation Files}

Use \texttt{//} for brief, inline, or explanatory comments within function implementations. These should clarify logic, document assumptions, or highlight non-obvious decisions.
\begin{lstlisting}[style=cppstyle]
// Apply default configuration values if not explicitly set
if (!settings.hasTimeout()) 
{
    settings.setTimeout(DEFAULT_TIMEOUT); // DEFAULT_TIMEOUT defined in config.h
}
\end{lstlisting}
\begin{lstlisting}[style=cppstyle]
// Bad: Sets timeout to default
settings.setTimeout(DEFAULT_TIMEOUT);

// Good: Use default timeout if none specified in config
settings.setTimeout(DEFAULT_TIMEOUT);
\end{lstlisting}
\textbf{Guidelines:}
\begin{itemize}\itemsep0em
    \item Place standalone comments on their own line above the code they describe.
    \item Use inline comments sparingly and only when the code is not self-explanatory.
    \item Do not restate what the code does—explain why it does it, if not obvious.
    \item Maintain clarity and brevity; comments should add value, not clutter.
\end{itemize}



\subsection{README Files and Higher-Level Documentation}

Each code directory must contain a \texttt{README} file that provides a concise overview of the directory’s purpose. The README should include:
\begin{itemize}\itemsep0em
    \item A brief description of the contents of the directory (key files, subfolders).
    \item The role and relationship of these files or components within the larger project.
    \item Any special instructions or notes relevant to developers navigating or modifying the directory.
\end{itemize}
This ensures quick orientation for developers and aids maintainability across the codebase.

At the project level, comprehensive documentation is maintained in the main MIA manual. This manual offers a high-level overview of:
\begin{itemize}\itemsep0em
    \item The complete set of features and functionality.
    \item Descriptions of libraries and core components.
    \item Available tools and utilities.
    \item Architectural design decisions and workflows.
\end{itemize}
The MIA manual serves as the primary reference for understanding the system holistically and should be kept up-to-date as the project evolves.

Together, directory-level README files and the MIA manual provide layered documentation: granular, file-level guidance paired with broad, conceptual understanding, facilitating both detailed development work and strategic planning.

\subsubsection*{Example README excerpt}

\begin{lstlisting}[style=cppstyle]
# apps Directory Overview

This directory contains the standalone apps for the MIA project. Below is a brief description of each subfolder:
  
- **template/**  
  Contains a fully functional app featuring most of the basic application framework techniques used in developing MIA apps. This is used to both test new features and provide a common template for storing correct usage of the various features.
\end{lstlisting}









\section{CMake Standards}

\subsection{Structure and Conventions}
A clean CMake structure improves build reproducibility and portability. Modern practices enable more maintainable and scalable build definitions. Clear separation of targets and dependencies reduces cross-target pollution.
\begin{itemize}\itemsep0em
    \item Use lowercase CMake commands (i.e., \texttt{add\_executable}, \texttt{target\_link\_libraries}).
    \item Group CMake files logically and modularly using subdirectories and \texttt{add\_subdirectory()}.
    \item Use modern CMake features (i.e., target\_include\_directories, target\_compile\_features).
    \item External dependencies must be managed using \texttt{FetchContent} or \texttt{find\_package}.
    \item Add comments to any blocks of cmake code which function together to perform a particular task. For example:
    \begin{lstlisting}[style=pythonstyle]
# Create the MIATemplate executable.
set(MIATemplate_SRC MIATemplate.cpp MIATemplate_main.cpp )
set(MIATemplate_INC MIATemplate.hpp )
add_executable(MIATemplate ${MIATemplate_SRC} ${MIATemplate_INC} )
target_link_libraries(MIATemplate PRIVATE Core_LIB )
    \end{lstlisting}
\end{itemize}











\section{Bash Scripting Standards}

\subsection{Style and Safety}
\begin{itemize}\itemsep0em
    \item Use \texttt{\#!/bin/bash} (or \texttt{\#!/usr/bin/env bash}) at the top of scripts.
    %\item Always use \texttt{set -euo pipefail} to catch errors early.
    \item Use functions to encapsulate logic (when appropriate) instead of writing top-level scripts.
    %\item Use \texttt{local} keyword inside functions for variable scoping.
    %\item Quote all variable references: \texttt{"\$\{var\}"}.
    %\item Prefer \texttt{\$(...)} over backticks for command substitution.
\end{itemize}

\subsection{Documentation and Naming}
\begin{itemize}\itemsep0em
    \item Start each script with a brief comment block describing its purpose and usage.
    \item Use descriptive function and variable names.
    \item Use comments to clarify non-obvious logic, especially when using subshells or traps.
\end{itemize}











\section{Enforcement and Exceptions}

\begin{itemize}\itemsep0em
    \item Continuous integration pipelines are still in development for this project which will enforce code quality.
    \item Code reviews are expected to evaluate adherence to these standards.
\end{itemize}

If a contributor proposes a deviation from these standards for good reason (i.e., performance, cross-platform needs), it must be documented and justified in the merge request. In all other cases, adherence is expected.





