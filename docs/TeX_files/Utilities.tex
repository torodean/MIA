\chapter{Utilities}
\label{chap:utilities}

This chapter documents reusable utility components that support common development needs across the project.












\section{Markov Model Generation Module}

\subsection{Overview}

In a previous project I was working on for Dungeons \& Dragons, I created a model for generating random names and character sequences. That project code can be found here: 
\begin{center}
\url{https://github.com/torodean/DnD/blob/main/templates/creator.py}. 
\end{center}
The functionality was based on transition patterns between characters of preexisting names. I only later found out that this was referred to as a Markov model. The features related to this which are being added to MIA are a more generalized version of that work, which will follow the naming convention relating them to Markov Models.

The \texttt{MarkovModels.hpp} header defines a templated C++ module for constructing and manipulating first-order discrete-time Markov models of arbitrary order. The implementation is generic and supports any comparable data type, including characters, strings, or custom state representations, etc.

\subsection{First-Order Markov Model}

This section contains an explanation of how the markov model functions. Consider some set of sequences $S$ for some arbitrary type, where each element is denoted by a capitalized letter, $S=\{ABC, ABD, BAD\}$. The probability matrix $P$ is constructed by determining all of the elements which follow another, and at what probability that element has of following the others (The probabilities of elements following some element $K$ is $P_K$). that is $P(S) = \{K:P_K \forall K \in S\}$.

The set of elements which exist in $S$ are ${A, B, C, D, \emptyset}$, where $\emptyset$ denotes the absence of an element (or beginning/end of a sequence). Starting with $A$, we can see that the $A$ element is followed only by $B$ (twice), and $D$ (once) in $S$. The total number of elements ever following an $A$ is thus three. The probabilities following an element $A$ is thus
\begin{align}
P_A = \begin{cases}
  B & :\text{twice} \\
  D & :\text{once}
\end{cases} \implies P_A = \begin{cases}
  B & : 66.\overline{6}\% \\
  D & : 33.\overline{3}\%
\end{cases} = \{B:0.\overline{6}, D:0.\overline{3}\}
\end{align}
Following this same process for the other elements gives
\begin{align}
P_B &= \{A:0.\overline{3}, C:0.\overline{3}, D:0.\overline{3}\} \\ P_C &= P_D = \{\emptyset:1.0\}\\ P_\emptyset &= \{A:0.\overline{6}, B:0.\overline{3}\}.
\end{align}
The total probability matrix for this set of sequences would then be
\begin{align}
P(S) = \{A:P_A, B:P_B, C:P_C, D:P_D, \emptyset: P_\emptyset\} = \begin{cases}
A:\{B:0.\overline{6}, D:0.\overline{3}\} \\
B: \{A:0.\overline{3}, C:0.\overline{3}, D:0.\overline{3}\} \\
C: \{\emptyset:1.0\} \\
D: \{\emptyset:1.0\} \\
\emptyset: \{A:0.\overline{6}, B:0.\overline{3}\}.
\end{cases}
\end{align}
The $\emptyset$ is a special case in that it represents the first character of a sequence (there is never a character after the last). This format may not look like a matrix at all, but it can be re-written to matrix format. First, note that there are a total of 5 elements ($A$, $B$, $C$, $D$, $\emptyset$) which will give a $5 \times 5$ matrix for all possible combinations. The matrix is configured such that both the rows and columns span from $A\rightarrow\emptyset$, covering all the elements of the set. The matrix value of $a, b$ then represents the probability that element $a$ will be proceeded by element $b$.
\begin{align}
P(S) = \left[
\begin{matrix}
0 & 0.\overline{3} & 0 & 0 & 0.\overline{6} \\ 
0.\overline{6} & 0 & 0 & 0 & 0.\overline{3} \\ 
0 & 0.\overline{3} & 0 & 0 & 0 \\ 
0.\overline{3} & 0.\overline{3} & 0 & 0 & 0\\ 
0 & 0 & 1.0 & 1.0 & 0 \\ 
\end{matrix}\right]
\end{align}
This probability matrix thus represents the probability of an element proceeding another in one of the given sequences. It can be used to generate new sequences which adhere to similar patterns of the input sequences. With larger data sets, more possibilities of sequences typically arise as probable outputs. 

One important feature of these models is that under low-entropy (the model is derived from a deterministic source), a uniquely resolvable input set (You can reconstruct exactly one input set) and with enough metadata (initial state, model size, model order, etc), the model can be used to reconstruct the original data.


\subsection{Functionality}

This module provides the following key features:

\begin{itemize}
    \item \textbf{Probability Matrix Construction:} 
    Generates a normalized transition matrix from a collection of input sequences, where transitions between adjacent elements are counted and then normalized into probabilities.

    \item \textbf{State Querying:}
    Provides utility functions to:
    \begin{itemize}
        \item Retrieve successor probabilities for a given state.
        \item Check whether a given state exists in the matrix.
        \item Extract the list of all states in the model.
    \end{itemize}

    \item \textbf{Matrix Utilities:}
    Includes functions to:
    \begin{itemize}
        \item Clear the matrix.
        \item Print the matrix to an output stream.
    \end{itemize}

    \item \textbf{Sampling Support (Planned):}
    A placeholder exists for a future function that will enable sampling the next state from a given current state based on its transition distribution.
\end{itemize}

\subsection{Design Notes}

The transition matrix is represented using nested \texttt{std::unordered\_map} containers. Specifically, the outer map keys are current states, and the values are maps from successor states to transition probabilities. The code leverages C++17 features, including structured bindings and template aliasing, and is contained within the \texttt{markov\_models} namespace for modularity and clarity.










